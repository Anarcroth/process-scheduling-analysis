\documentclass{article}

\usepackage{graphicx}
\usepackage{amsmath}
\usepackage{color}
\usepackage{fullpage}

\renewcommand{\familydefault}{\sfdefault}

\definecolor{codegray}{gray}{0.9}
\newcommand{\code}[1]{\colorbox{codegray}{\texttt{#1}}}

\title{Senior Project Thesis\\
       \large Process scheduling - comparison and contrast}
\date{\today}
\author{Martin Nestorov}
\linespread{1}

\begin{document}

\maketitle
\pagenumbering{arabic}

\newpage

\section{Introduction}

Every Operating System has some type of process handling capabilities, be that in the form of simple queue structure, or in some complex algorithm. This is also specific to the different types of systems that are handling the jobs. Some embedded systems do not have the capacity to handle complex operations, which forces them to have simple scheduling algorithms. One such example would be preemptive OSes.

There are several process scheduling algorithms that are used in batch, interactive, and real-time systems. These include, but are not limited to, First Come First Serve (\textbf{FCFS}), Shortest Job First (\textbf{SJF}), Priority Scheduling, Round-Robbin Scheduling, Guaranteed Scheduling, Lottery Scheduling, and Multilevel Queue Scheduling. All of them have their advantages and weaknesses. Some are simpler and work for small sequential systems, while others are more complex, but distribute the workload better.

The purpose of this project is to analyze and compare these different algorithms, to show their strengths and weaknesses.

Another thing to consider is the type of the system that lies under the processes. In general, we can either consider a \textit{real-time} system, or an \textit{interactive} one. \textit{Real-time} systems are such that take into consideration time as an essential role. Typically, one or more devices can stimulate the system and it has to react accordingly in a certain amount of time. \textit{Interactive} systems, much like the 'real-time' ones, can and are stimulated by other programs, but don't have such a strict time constraint.

\section{Specification and Analysis of the Software Requirements}

\section{Design of the Software solution}

This software uses several algorithms and data-structures that play a key role in the whole inner-workings. Because the purpose of the project is to show how different algorithms affect the process execution of real-time systems, we have to talk about each used algorithm and the accompanying data-structures used.

\subsection{Algorithms and Data-structures}

\textbf{FCFS}

The \textbf{F}irst \textbf{C}ome \textbf{F}irst \textbf{S}erve algorithm is one of the easiest to understand and implement. It can be looked at from many different angles. \textbf{FCFS} can be seen as a \code{linked-list} or a \code{queue}, which just serves each incoming process to the CPU for execution.

\section{Implementation}

This software is developed entirely with the \code{C++} programming language. All of the implementation uses the \code{C++ 11} standard (and up). The \code{C++ 11} standard and all of its follow-up additions, all the way up to \code{C++ 20}, have great benefits to creating modern, safe, and easy to manage software. Many functionalities introduced since \code{C++ 11} have been used to create this project. Most noticeably, the use of lambda functions, collection manipulators, threads and mutexes, and many more, play a key role.

In addition to this, the famous library \code{ncurses} is used in order to make working with terminal emulators easier. \code{ncurses} provides an \code{API} for manipulating the graphics and output of the terminal. Since this is an application, based on working with a terminal emulator, such a library would be of great help. The specific terminal that was used to test and run the application is \code{xterm}, but this was also tested on \code{gnome-terminal}.

The operating system used to create the software is \code{Arch Linux}, with additional testing environment under \code{Fedora 29}.

\section{Testing}

\section{Result and Conclusion}

\section{References}

\end{document}